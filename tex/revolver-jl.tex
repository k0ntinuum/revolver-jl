

\documentclass{article}
\usepackage[utf8]{inputenc}
\usepackage{setspace}
\usepackage{ mathrsfs }
\usepackage{amssymb} %maths
\usepackage{amsmath} %maths
\usepackage[margin=0.2in]{geometry}
\usepackage{graphicx}
\usepackage{ulem}
\setlength{\parindent}{0pt}
\setlength{\parskip}{10pt}
\usepackage{hyperref}
\usepackage[autostyle]{csquotes}

\usepackage{cancel}
\renewcommand{\i}{\textit}
\renewcommand{\b}{\textbf}
\newcommand{\q}{\enquote}
%\vskip1.0in



\begin{document}

\begin{huge}



{\setstretch{0.0}{

\b{Revolver} is just Prefix with an extra feature, which is to say a symmetric cryptosystem based primarily on prefix/instantaneous codes.  Since the union of prefix codes is not in general itself a prefix code, this makes the encrypted output difficult to tokenize. Please see the pdf in \b{prefix-jl} repo to understand Prefix.

What Revolver adds is an evolution of the Prefix key that maintains its functionality. Essentially an entire mode's prefix code is \q{rolled} forward using both the plaintext and ciphertext so that it remains a prefix code. Its symbols are permuted. Actually the input code (not a prefix code) is also permuted, for a different but also compatible effect.  This small tweak might (intuitively seems that it should) make the same key far more secure. 


}}
\end{huge}
\end{document}
